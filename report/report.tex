\documentclass[a4paper,12pt]{scrartcl}
\usepackage[utf8]{inputenc}
\usepackage{graphicx}
\usepackage{mathtools}
\usepackage{amsfonts}
\usepackage{gnuplottex}
\usepackage{fullpage}
\usepackage{wrapfig}
\usepackage{hyperref}
\usepackage{url}

% \setcounter{secnumdepth}{3}

\DeclareMathOperator*{\argmax}{arg\,max}
\DeclareMathOperator*{\argmin}{arg\,min}

\begin{document}

\titlehead{Université Libre de Bruxelles}
\title{Procedural City Generator Project}
\subtitle{INFO-H-502 - Image synthesis}
\author{Tim Lenertz}
\date{\today}
\maketitle

\section{Introduction}
The goal of this project is to develop a Blender add-on that generates a randomized city model. The city consists of a street layout, different kinds of buildings, lakes, gardens and parcs. All of these elements should be randomized as well.

\section{Methodology}
In order to generate a credible model of a random city, a logical approach is to emulate the way a city evolves in reality: Streets are shaped according to the shape of the terrain, and in relation to previously existing infrastructure. Also the most developped part tends to be located near a city center, while more remote areas remain more rural.

This project is based on the \emph{Citygen} system described in \cite{Kell2007}. The city model is generated following these steps:
\begin{enumerate}
\item A terrain height-map is created featuring some erosion.
\item Primary streets are put on the terrain, and made to connect some randomly defined intersection points. The streets' shape is guided by the terrain shape.
\item The regions enclosed by primary streets are the \emph{city cells}. According to its distance from the center, each city cell is attributed a \emph{profile} that indicates whether this cell will contain a given type of city buildings and streets, or more rural content.
\item For city cells that shall contain city buildings, a network of \emph{secondary roads} is created. Starting from two or more points on the enclosing primary street cycle, secondry roads are \emph{grown} into the city cell using a recursive algorithm that simulates the real evolution of a street layout.
\item Each region enclosed by these secondary roads is called a \emph{block}. Each block is subdivided into \emph{lots}, and the lots placed next to a road will contain a building.
\item For each of these lots, a certain kind of building is created with some randomization.
\end{enumerate}
The following text describes these 6 steps in more detail. The challenged encountered in implementing the algorithms are described in section \ref{sec:implementation}.

\subsection{Terrain}

\subsection{Primary Streets}

\subsection{City Cells}

\subsection{Secondary Roads}

\subsection{Blocks}

\subsection{Buildings}

\section{Implementation}
\label{sec:implementation}

\section{Results}


\bibliographystyle{authordate1}
\nocite{*}
\bibliography{../references/references.bib}



\end{document}